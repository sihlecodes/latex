\chapter{Introduction}
This chapter provides an overview of the content structure of the study and the background of the issues underlying this study. The research objectives are comprehensively stated in accordance with the research outputs the study seeks to achieve. The motivations behind the present study are clearly stated. Moreover, an overview of the format/layout of the entire dissertation is briefly discussed and the content of the following chapter is stated. 

\section{Overview}
Airports Company South Africa (ACSA) is a main airport management enterprise in South Africa. ACSA was established in 1993 as a public company under the Airport Act (No. 44, 1993). Most of ACSA is owned by the South African government. Nevertheless, ACSA is active in commercial law. Over the years, the company has succeeded in providing a customer-oriented, commercially successful company whose airport has proven to be a symbolic and vital service element of South Africa. To date, ACSA owns and manages nine (9) South African airport communities in several provinces. As a state-owned enterprise, ACSA has an extra mandate to sincerely hand over profitability to its shareholders.
The company owns a number of assets that are critical thus should be effectively and efficiently maintained. As a consequence, ‘fire protection’ is a critical sector within any aviation industry, which has created divergent opinions in terms of compliance standards. Aircraft accidents are devastating, considering that loss of lives and costly equipment must be expected. In aviation, firefighting foam, particularly aqueous film-forming foam (AFFF) is a sole optimum extinguishing agent for suppression of combustible or flammable liquids. In South Africa, aviation accidents involving aircrafts have dramatically decreased over the past years. Nonetheless, it is obligatory for ACSA to adhere to the relevant compliance standards and be fully equipped in case of any unexpected circumstances.
Periodic training is mandatory in any aviation fire protection in ensuring that firefighting skills and resources within the sector adhere to the Federal Aviation Administration (FAA), National Aviation Authority (NAA), and National Fire Protection Association (NFPA) compliance standards, to react rapidly during accidents. Consequently, periodic testing of the performance parameters of fire extinguishing foam is a necessity. Unexpected circumstances often happen during periodic tests, when AFFF is unable to perform as anticipated. The main priority of AFFF is to suppress the fire and allow possible victims to have more time to escape during the accident. However, all this must be achieved in 1 minute or less upon the arrival on the accident scene, according to relevant compliance standards. The poor performance of AFFF is caused by numerous and diverse factors, hence it becomes difficult to examine where the problem originates.
In recent years, there have been fatal fire accidents in the aviation industry, which has tasked researchers to investigate further on the fire protection sector. Nevertheless, there are still notable gaps in previous research conducted, with limited studies on the impact of the storage tank of the firefighting foams. This is due to the nature and diversity of these problems. The complexity in focusing on the optimization of firefighting foam on the storage facility has always been a concern due to complicated branches of engineering such as material sciences, structural analysis, and thermal engineering involved. 
In 1965 Meldrum et al \cite{meldrum1965storage} published their study on storage life and utility of mechanical firefighting foam liquids. The study contributed to predicting the period in which firefighting foams can be held in a storage facility before they deteriorate. Furthermore, it emphasized the gaps and limitations within previous research and the difficulties in problem optimization. The predictions are governed by parameters such as extreme temperatures, oxidation, evaporation, corrosion, dilution, and contamination on the storage facility. Most of this research work will be benchmarked by \cite{meldrum1965storage}, extensive research on the impact the storage facility has on AFFF will be conducted in various branches of engineering (material sciences, structural analysis, and thermal engineering).
This research work has a great significance in terms of evaluating the impact of storage facilities/tanks on AFFF performance, which is also dependent on a number of factors that have been previously stated. All manufacturers of firefighting foam concentrate have strict recommendations of storing their products, with the priority being to store foam concentrate in its original storage tank. The challenge rises due to critical factors that must be taken into consideration and often result to large storage tanks being constructed on-site. The large storages are beneficial as foam concentrate can be pumped rapidly from one source to firefighting vehicles during emergency conditions without the huge demand of replenishing. Consequently, these critical considerations lead to storing firefighting foam concentrate in different storage tanks rather than the recommended original containers.
 Motivated by these challenges, the purpose of the present research work is to experimentally evaluate and assess the impact of the materials utilised storage facility on the performance parameters of AFFF (state the exact parameters we are concerned with). The study examines the compatibility of engineering materials such as high density polyethylene (HDPE), mild steel and stainless steel. These are the materials that are commonly utilised when constructing a storage facility for firefighting concentrate, AFFF in particular. In this way, it is possible to enhance the properties of these engineering materials based on the heat treatment process, microstructural analysis, environmental stress cracking, and corrosion phenomenon. All these optimization methods are aiming to effectively store AFFF solution without the affection of the performance parameters and chemical composition during vital fire accidents. 

\section{Thesis layout}
The study consists of six chapters structured as follows:

\noindent \textbf{Chapter 1: Introduction} \\ 
This chapter provides a brief overview of the organisation included in the study as well as the background to the problem underlying the research work. It outlines the aim and the importance of the study. It explains the research objectives, formulates the research questions, and discusses the format of the study.

\noindent \textbf{Chapter 2: Literature review} \\
Chapter 2 provides a comprehensive summary of the existing literature on AFFF that was consulted. It focuses on the broader knowledge of previous studies that relate to the problems of interest in this study.

\noindent \textbf{Chapter 3: Evaluation of materials of construction} \\
The current and previous studies on the compatibility of engineering materials with AFFF concentrate is conducted. This is then validated by the experimental analysis in chapter 6 based on the outcomes.

\noindent \textbf{Chapter 4: Experimental set up} \\
The experimental set-up for the tests that were conducted in this study are documented in this chapter. All the materials and equipment used during the tests are also shown.

\noindent \textbf{Chapter 5: Results and discussion} \\
This chapter documents and discusses the results from the experimental tests in chapter 5. The results are compared between the various engineering materials. Results from other researchers are also compared with that found in this study. 

\noindent \textbf{Chapter 6: Conclusions and further work} \\
This chapter concludes the study with a summary of research findings aimed at solving the
research problem. The scope for future research areas is discussed and the conclusion for the entire study made.