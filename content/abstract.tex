\section*{Abstract}
\addcontentsline{toc}{chapter}{\bf Abstract}


\Acrfull{afff} has become a critical constituent within the aviation industry. It was declared by \Acrfull{nfpa} as the sole optimum extinguishing agent for the suppression of hydrocarbon fires. Nevertheless, unexpected circumstances occur when \acrshort{afff} is unable to perform as anticipated. In the aviation industry, this is normally noticed during periodic tests. As a consequence, it has become a necessity to monitor the performance of \acrshort{afff} regularly. 

This study aims to investigate the impact of the storage facility on the performance parameters of \acrshort{afff}. The principal focus is on the materials that are commonly used to construct the storage tanks used for storing \acrshort{afff} concentrate, namely mild steel, stainless steel, and \acrfull{hdpe}. The aim focuses specifically on ascertaining if these materials affect the properties of \acrshort{afff} concentrate, thus influencing its performance during firefighting. In this way, the optimization of these materials in such a way that they are compatible with \acrshort{afff} concentrate becomes a feasible alternative. 

A qualitative research approach was used to investigate the impact of the materials used to construct the storage tank for storing \acrshort{afff} concentrate. Based on the aforementioned challenges, Fourier transforms infrared spectroscopy (\acrshort{ftir}), \acrfull{tem}, dynamic light scattering, and \acrfull{icp-aes} were all used for testing the materials. The findings from the primary research concluded that the three materials affect the foaming ability and foam stability of the \acrshort{afff} solution, with the severity being the variance. \acrshort{hdpe} was found to have a less severe impact, while stainless steel had a tolerable impact. The findings demonstrated that mild steel greatly affects the aforementioned performance parameters of \acrshort{afff}. Based on these findings, it was concluded that mild steel is incompatible with \acrshort{afff} concentrate. 

The study was limited to the effects of the materials on any \acrshort{afff} performance parameters. The study provided sufficient relevant literature on the three materials to thoroughly comprehend the behaviour of these materials. Consequently, the study recommends that mild steel should be heat treated to enhance its properties in such a way that it is innocuous to \acrshort{afff} concentrate. The study further recommends that other materials, such as fibreglass and \acrfull{xlpe}, be established as potential materials for constructing \acrshort{afff} storage facilities.
