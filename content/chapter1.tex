\chapter{Introduction}
\section{Introduction}
This chapter provides an overview of the research content structure and the background to the questions underlying this study. The aims of the study are exhaustively outlined in terms of the research outcomes it seeks to achieve. The motivations for this study are clearly stated. In addition, the format/layout of the entire thesis is briefly discussed, and the content of the next chapter is outlined.

\section{Background to the Problem}
Fire protection is a critical sector in any aviation industry, which has given rise to differing opinions on compliance standards. Aviation accidents are devastating, given the loss of life and costly equipment to be expected. In aviation, fire-fighting foam, especially \Acrfull{afff}, is the only optimum means of extinguishing flammable or combustible liquids. In fact, South Africa has seen a dramatic decrease in aircraft accidents in recent decades. Nevertheless, the aviation industry must adhere to appropriate compliance standards and be fully equipped for any contingency. 

Periodic training is mandatory in all aspects of aviation fire protection to ensure that the industry's firefighting skills and resources meet \Acrfull{faa}, \Acrfull{naa} and \Acrfull{nfpa} standards for rapid response during accidents. Consequently, periodic testing of extinguishing foam performance parameters is a necessity. During periodic testing, unforeseen circumstances often arise where the \acrshort{afff} cannot perform as expected. The main priority of \acrshort{afff} is to suppress the fire and give possible victims more time to escape during an accident. However, according to relevant compliance standards, all of this must be accomplished in one minute or less after arrival at the accident scene. The low effectiveness of \acrshort{afff} is caused by numerous and varied factors, so it becomes difficult to determine where exactly the problem originates.

\section{Significance of the Study}
Fatal fires have occurred in the aviation industry in recent decades, prompting researchers to further study the fire protection sector. Nevertheless, there are still notable gaps in previous research, in particular, the effects of the firefighting foam storage tank have been little studied. This is due to the nature and diversity of these problems. The difficulty of focusing on the optimization of fire foam in storage has always been a concern because of the complex engineering branches such as materials science, structural analysis, and thermal engineering.  

\cite{meldrum1965storage} published their study on the shelf life and usefulness of mechanical firefighting foam fluids. The study was instrumental in predicting the period for which firefighting foam can be stored before it deteriorates. It also highlighted gaps and limitations in previous studies and difficulties in optimizing the problem. Predictions depend on parameters such as temperature extremes, oxidation, evaporation, corrosion, dilution, and contamination of the storage facility. Much of this research work will be compared to \cite{meldrum1965storage}. Extensive research will be done on the effects of storage or reservoir on the \acrshort{afff}. 

This research work is important to evaluate the effect of storage facilities on \acrshort{afff} performance, which also depends on several of the factors mentioned above. All firefighting foam concentrate manufacturers have strict guidelines for the storage of their products, with storage of foam concentrate in the original tank being a priority. The difficulty arises because of critical factors that must be taken into consideration and often leads to the construction of large on-site storage tanks. Large storage tanks are advantageous because the foam concentrate can be quickly pumped from one source to the fire trucks during emergencies without the need for restocking. Consequently, these critical considerations lead to storing foam concentrate for firefighting in a variety of tanks rather than the original, recommended containers.

\section{Objectives and Methodology}
Motivated by these challenges, the purpose of the present research work is to experimentally evaluate and assess the impact of the storage facility on any performance parameters of \acrshort{afff}. The study examines the compatibility of engineering materials, namely mild steel, stainless steel, and \acrfull{hdpe}. These are the materials that are commonly used when constructing a storage facility for firefighting foam concentrate, specifically \acrshort{afff}. This is accomplished by performing several analyses that evaluate the effect of each material on the \acrshort{afff} concentrate and, thus, on the performance parameters. In all these analyses, the pure \acrshort{afff} concentrate is utilized as a benchmark to deduce any vital alterations. In such a manner, it is possible to enhance the properties of these engineering materials based on the heat treatment process, microstructural analysis, and environmental stress cracking in such a way that they are compatible with \acrshort{afff} concentrate. All these optimization methods aim to effectively store \acrshort{afff} concentrate without affecting its performance parameters or chemical composition during vital fire accidents.  

\section{Scope}
The scope of this study is limited in the following regards: Firstly, the study is only concerned with the effect of the materials on \acrshort{afff} concentrate. It does not investigate the causes of these effects, but this can be accomplished in future research. Secondly, the study does not practically test the performance of \acrshort{afff}, as would normally happen in the aviation industry during periodic tests or aircraft accidents. Rather, the study experimentally assesses the properties of \acrshort{afff} concentrate and then draws scientific conclusions based on the relevant literature.

\section{Thesis layout}
The study consists of six chapters structured as follows:

\subparagraph*{Chapter 1: Introduction}\hfill\\
This chapter provides a brief overview of the organisation included in the study as well as the background to the problem underlying the research work. It outlines the aim and importance of the study. It explains the research objectives, formulates the research questions, and discusses the format of the study.

\subparagraph*{Chapter 2: Literature review}\hfill\\
Chapter 2 provides a comprehensive summary of the existing literature on \acrshort{afff} that was consulted. It focuses on the broader knowledge of previous studies that relate to the problems of this study.

\subparagraph*{Chapter 3: Evaluation of materials of construction}\hfill\\
The current and previous studies on the compatibility of engineering materials with \acrshort{afff} concentrate have been conducted. This is then validated by the experimental analysis in Chapter 5 based on the outcomes. 

\subparagraph*{Chapter 4: Experimental set up}\hfill\\
The experimental set-up for the tests and analyses that were conducted in this study is documented in this chapter. All the materials and equipment used during the tests are also shown, and their significance is concisely described.  

\subparagraph*{Chapter 5: Results and discussion}\hfill\\
This chapter documents and discusses the results of the experimental tests conducted in Chapter 4. The results were recorded and benchmarked with the standard parameters. The conclusions are drawn based on outcomes and the existing literature review.  

\subparagraph*{Chapter 6: Conclusions and further work}\hfill\\
This chapter concludes the study with a summary of research findings aimed at solving the research problem. The scope for future research areas is discussed, and the conclusion for the entire study is made. 

\section{Conclusion}
Considering the devastation caused by hydrocarbon fires, the fire protection sector cannot be overlooked in the aviation industry. This chapter provided a brief overview of the entire study, which included background information on the problem, a statement of the problem, objectives, and the significance of the study. The format of the study was also described. The next chapter explores the literature that is relevant to this study.