\chapter{Abstract}
Aqueous Film Forming Foam (AFFF) has become a critical constituent within the aviation industry. It was declared by National Fire Protection Association (NFPA) as a sole optimum extinguishing agent for suppression of hydrocarbon fires. Nevertheless, unexpected circumstances occur when AFFF is unable to perform as anticipated. In the aviation industry, this is normally noticed during periodic tests. As a consequence, it has become a necessity to monitor the performance of AFFF on a regular basis.
The aim of this study is to investigate the impact of the storage facility on the performance parameters of AFFF. The principal focus is on the materials that are commonly used to construct the storage tanks used for storing AFFF concentrate, namely mild steel, stainless steel, and high-density polyethylene (HDPE). The aim focuses specifically on ascertaining if these materials affect the properties of AFFF concentrate, thus influencing its performance during firefighting. In this way, it is uncomplicated to explore materials that are innocuous to AFFF concentrate. Furthermore, the optimization of these materials in such a way that they are compatible with AFFF concentrate becomes a feasible alternative.
A qualitative research approach was used to investigate the impact of the materials used to construct the storage tank for storing AFFF concentrate. Based on the aforementioned challenges, Fourier transform infrared spectroscopy (FTIR), transmission electron microscopy (TEM), dynamic light scattering, and inductively coupled plasma atomic emission spectroscopy (ICP-AES) were all used for testing the materials. The findings from the primary research concluded that the three materials affect the foaming ability and foam stability of the AFFF solution, with the severity being the variance. HDPE was found to have a less severe impact, while stainless steel had a tolerable impact. The findings demonstrated that mild steel greatly affects the aforementioned performance parameters of AFFF. Based on these findings, it was concluded that mild steel is incompatible with AFFF concentrate.
The study was limited to the effects of the materials on any AFFF performance parameters. Consequently, the study recommends that mild steel should be heat treated to enhance its properties in such a way that it is innocuous to AFFF concentrate. The study further recommends that other materials, such as fiber glass and cross-linked polyethylene (XLPE), be established as potential materials for constructing AFFF storage facilities.