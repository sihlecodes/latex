To date, there are three (3) methods of generating firefighting foam from a foam solution namely: aspirated nozzle, Compressed air foam (CAF), and chemical reaction method (\cite{laundess2012suppression}). This technique is more useful for aviation fire protection due to the type of environment and aviation standards that were developed and led by the National Fire Protection Association (NFPA). Technically and according to the research, the aspirated nozzle is suitable for low expansion foams such as AFFF and AR-AFFF (\cite{xi2017experimental}).

The CAF method is commonly used for generating any kind of firefighting foam and was developed initially by the National Research Centre of Canada (NRCC) in the late 1990s (\cite{rie2016class}). There are numerous methods they have been extensively utilized to characterize the defects in steels. However, according to some researchers, the Transmission Electron Microscopy (TEM) technique is more suitable to characterize defects on steels (\cite{george2002introduction, bhadeshia2017steels}). Besides, the TEM method has been proven to be reliable on numerous occasions.

Over the past years, there have been several efforts made in order to avoid ESC. Therefore, using an appropriate resin formulations of ESCR materials, designing the geometric appropriately, carefully using the manufacturing controls that prevents occurrence of severe stress risers, and limiting stresses and strains during the storage facility installation, all these are usually sufficient to avoid ESC (\cite{gabriel1998history}). Moreover, PE polymer may be cross-linked to improve the chemical properties and thus resist cracking.  With this regard, it is vital to test the compatibility of PE, especially XLPE with AFFF concentrate in order to avoid unexpected circumstances during fire conditions. To date, there are over 40 different ESCR test methods that are used to determine the chemical resistance of various materials. The standard test that is currently used in the industry of polyethylene is bent-strip test (\cite{gabriel1998history}). The method is normally used to assess the performance of polyethylene cable insulation but can be cautiously used to evaluate the performance of XLPE storage facilities in the presence of AFFF concentrate. The bent-strip test is shown in Figure \ref{ch3:figure:bending_apparatus}. Where the specimen is immersed into a surfactant of interest, and the time to failure is noted. The results are reported using the notation $F_{xx}$, where xx is the percentage of samples that has been tested.

Originally, five types of firefighting foams are commonly used: FLUORO PROTEIN FOAMs (FPs), aqueous film-forming foams (AFFFs), film-forming fluoro protein foams (FFFPs), alcohol-resistant aqueous film-forming foams (AR-AFFFs) and Alcohol-resistant film-forming fluoro protein foams (AR-AFFFPs)