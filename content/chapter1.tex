\chapter{Introduction}
\section{Introduction}
This chapter provides an overview of the content structure of the study and the background of the issues underlying this study. The research objectives are comprehensively stated in accordance with the research outputs the study seeks to achieve. The motivations behind the present study are clearly stated. Moreover, an overview of the format/layout of the entire dissertation is briefly discussed, and the content of the following chapter is stated.

\section{Background to the Problem}
Fire protection is a critical sector within any aviation industry, which has created divergent opinions in terms of compliance standards. Aircraft accidents are devastating, considering the loss of lives and costly equipment that must be expected. In aviation, firefighting foam, particularly aqueous film-forming foam (AFFF), is the sole optimum extinguishing agent for suppression of combustible or flammable liquids. As a matter of fact, in South Africa, aviation accidents involving aircraft have dramatically decreased over the past decades. Nonetheless, it is obligatory for the aviation industry to adhere to the relevant compliance standards and be fully equipped in case of any unexpected circumstances.

Periodic training is mandatory in all aspects of aviation fire protection in ensuring that firefighting skills and resources within the sector adhere to the Federal Aviation Administration (FAA), National Aviation Authority (NAA), and National Fire Protection Association (NFPA) compliance standards, to react rapidly during accidents. Consequently, periodic testing of the performance parameters of fire extinguishing foam is a necessity. Unexpected circumstances often happen during periodic tests when AFFF is unable to perform as anticipated. The main priority of AFFF is to suppress the fire and allow possible victims more time to escape during the accident. However, according to relevant compliance standards, all of this must be accomplished in one minute or less upon arrival at the accident scene. The poor performance of AFFF is caused by numerous and diverse factors; hence, it becomes difficult to examine where the problem originates.

\section{Significance of the Study}
In past decades, there have been fatal fire accidents in the aviation industry, which has tasked researchers to investigate further on the fire protection sector. Nevertheless, there are still notable gaps in previous research conducted, with limited studies on the impact of the storage tank of the firefighting foams. This is due to the nature and diversity of these problems. The complexity of focusing on the optimization of firefighting foam in the storage facility has always been a concern due to the complicated branches of engineering such as material sciences, structural analysis, and thermal engineering involved. 

In 1965, Meldrum et al. \cite{meldrum1965storage} published their study on the storage life and utility of mechanical firefighting foam liquids. The study contributed to predicting the period in which firefighting foams can be held in a storage facility before they deteriorate. Furthermore, it emphasized the gaps and limitations within previous research and the difficulties in problem optimization. The predictions are governed by parameters such as extreme temperatures, oxidation, evaporation, corrosion, dilution, and contamination on the storage facility. Most of this research work will be benchmarked by \cite{meldrum1965storage}. Extensive research on the impact the storage facility or tank has on AFFF will be conducted.

This research work has great significance in terms of evaluating the impact of storage facilities on AFFF performance, which is also dependent on a number of factors that have been aforementioned. All manufacturers of firefighting foam concentrate have strict recommendations for storing their products, with the priority being to store the foam concentrate in its original storage tank. The challenge arises due to critical factors that must be taken into consideration and often result in large storage tanks being constructed on-site. The large storage facilities are beneficial as foam concentrate can be pumped rapidly from one source to firefighting vehicles during emergency conditions without the huge demand for replenishment. Consequently, these critical considerations lead to storing firefighting foam concentrate in different storage tanks rather than the original, recommended containers.

\section{Objectives and Methodology}
Motivated by these challenges, the purpose of the present research work is to experimentally evaluate and assess the impact of the storage facility on any performance parameters of AFFF. The study examines the compatibility of engineering materials, namely mild steel, stainless steel, and high density polyethylene (HDPE). These are the materials that are commonly used when constructing a storage facility for firefighting foam concentrate, specifically AFFF. This is accomplished by performing several analyses that evaluate the effect of each material on the AFFF concentrate and, thus, on the performance parameters. In all these analyses, the pure AFFF concentrate is utilized as a benchmark to deduce any vital alterations. In such a manner, it is possible to enhance the properties of these engineering materials based on the heat treatment process, microstructural analysis, and environmental stress cracking in such a way that they are compatible with AFFF concentrate. All these optimization methods aim to effectively store AFFF concentrate without affecting its performance parameters or chemical composition during vital fire accidents.

\section{Scope}
The scope of this study is limited in the following regards: Firstly, the study is only concerned with the effect of the materials on AFFF concentrate. It does not investigate the causes of these effects, but this can be accomplished in future research. Secondly, the study does not practically test the performance of AFFF, as would normally happen in the aviation industry during periodic tests or aircraft accidents. Rather, the study experimentally assesses the properties of AFFF concentrate and then draws scientific conclusions based on the relevant literature.

\section{Thesis layout}
The study consists of six chapters structured as follows:

\subparagraph*{Chapter 1: Introduction}\hfill\\
This chapter provides a brief overview of the organisation included in the study as well as the background to the problem underlying the research work. It outlines the aim and importance of the study. It explains the research objectives, formulates the research questions, and discusses the format of the study.

\subparagraph*{Chapter 2: Literature review}\hfill\\
Chapter 2 provides a comprehensive summary of the existing literature on \acrshort{afff} that was consulted. It focuses on the broader knowledge of previous studies that relate to the problems of interest in this study.

\subparagraph*{Chapter 3: Evaluation of materials of construction}\hfill\\
The current and previous studies on the compatibility of engineering materials with AFFF concentrate have been conducted. This is then validated by the experimental analysis in Chapter 5 based on the outcomes.

\subparagraph*{Chapter 4: Experimental set up}\hfill\\
The experimental set-up for the tests and analyses that were conducted in this study is documented in this chapter. All the materials and equipment used during the tests are also shown, and their significance is concisely described.

\subparagraph*{Chapter 5: Results and discussion}\hfill\\
This chapter documents and discusses the results from the experimental tests conducted in Chapter 4. The results were recorded and benchmarked with the standard parameters. The conclusions are drawn based on outcomes and the existing literature review. 

\subparagraph*{Chapter 6: Conclusions and further work}\hfill\\
This chapter concludes the study with a summary of research findings aimed at solving the research problem. The scope for future research areas is discussed and the conclusion for the entire study made.

\section{Conclusion}
Considering the devastation caused by hydrocarbon fires, the fire protection sector cannot be overlooked in the aviation industry. This chapter provided a brief overview of the entire study, which included background information on the problem, a statement of the problem, objectives, and the significance of the study. The format of the study was also described. The next chapter explores the literature that is relevant to this study.