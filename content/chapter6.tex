\chapter{Conclusions and Further Studies}
\label{ch6:anchor:chapter}
\section{Introduction}
This chapter presents the conclusion and recommendations for future work on this study. The impact of using mild steel, stainless steel, and \acrshort{hdpe} as storage facilities for \acrshort{afff} concentrate was analyzed. This was achieved by performing various tests on the materials and, therefore, conducting thorough analyses. The relevant literature review and findings were used to draw conclusions and make recommendations for future studies.

\section{Findings from the Primary Research}
Findings from the primary research were presented in relation to research questions that this study strived to answer.

\begin{itemize}
    \item The functional group of C-F stretch shifted when mild steel, stainless steel, and \acrshort{hdpe} were immersed. The largest shift was observed when the \acrshort{afff} concentrate interacted with mild steel. The elemental analyses further support this reaction.
    \item In a pure state, \acrshort{afff} concentrate possesses a single crystalline structure, whereas when the three materials were immersed, it critically changed to a polycrystalline structure. Most of the crystal structural changes occurred when \acrshort{afff} concentrate reacted with mild steel.
    \item The spherical particle shape of \acrshort{afff} concentrate was not altered by reacting with mild steel, stainless steel, or \acrshort{hdpe}. This implies that the \acrshort{afff} concentrate can retain its viscosity after reacting with the three materials.
    \item The diameter of the particles in the pure \acrshort{afff} concentrate was altered when it reacted with the materials of interest. The particle size had a percentage increase of 36.419\% and 12.549\% when stainless steel and \acrshort{hdpe} were immersed, respectively. On the other hand, when mild steel was immersed, the particle size increased immensely by 655.808 nm. Based on this, it was discovered that the reaction of mild steel with \acrshort{afff} concentrate reduces the diffusion rate of fluorosurfactant molecules, which has a significant impact on \acrshort{afff}'s foaming ability. 
    \item Narrow peaks observed in the PSD when \acrshort{afff} concentrate reacts with stainless steel or \acrshort{hdpe} do not have a vital impact on the diffusion rate of fluorosurfactant molecules. It further demonstrated that the two materials do not influence the spreading ability of \acrshort{afff} during firefighting.
    \item When AFFF concentrate reacted with mild steel, the PSD revealed a wide peak. This demonstrated that the spreading ability of AFFF might have been slightly reduced. This suggests that the alteration of PSD can cause a slight impact on the spreading capability of any aqueous concentrate \cite{machhi2021effect}.
    \item The elemental analyses demonstrated that the reaction of the \acrshort{afff} concentrate with the materials of interest reduced the sulphur and sodium content of the concentrate. Consequently, this increases the surface tension of water with the \acrshort{afff} solution and thus decreases the stability of foam during firefighting. However, the \acrshort{ftir} spectra analyses confirmed the presence of N-C-S when \acrshort{afff} concentrate reacted with stainless steel and \acrshort{hdpe}, but it did not appear when it reacted with mild steel.
    \item The fluorine element was reduced in the \acrshort{afff} concentrate. This was confirmed by the \acrshort{ftir} spectra when observing the C-F stretch. This alteration is very harmful, as it reduces the blanketing capabilities of \acrshort{afff} during firefighting.
    \item There was a massive degradation or wearing of mild steel when it interacted with \acrshort{afff} concentrate. This was caused by the critical increase of the iron element during the chemical reaction. Additionally, the elemental report showed that stainless steel and \acrshort{hdpe} also underwent the degradation process, but it was far less severe than in mild steel.
    \item The visual observation during the post-experimental work showed that the degradation of the materials influences the pureness of \acrshort{afff} concentrate, thus possibly influencing the fighting capabilities.
\end{itemize}

\section{Concluding Remarks}
Based on the findings, it is clear that mild steel is not compatible with the \acrshort{afff} concentrate. Although it is a relatively cheap option when selecting a storage tank for \acrshort{afff} concentrate, some alterations within the tank should be made to avoid the degradation of \acrshort{afff}. Stainless steel could be an option as well, as it has minimal effects on the \acrshort{afff}. The huge concern is that it is not economical, making it an unfeasible option. As a result, \acrshort{hdpe} is a viable option for storing \acrshort{afff} concentrate. However, as aforementioned, it has the setback of suffering from ESC. This is normally avoided by cross-linking the \acrshort{hdpe} to produce \acrfull{xlpe}. Nonetheless, more research should be conducted to validate its resistance to ESC. In addition, the plastic materials have the fundamental advantage of being inexpensive.
In the 21st century, the storage tanks are constructed using fiber glass. This material is gradually replacing the materials used for constructing the storage tanks due to several benefits, including being resistant to numerous chemicals \cite{avdeeva2016chemical}. As a consequence, fiber glass is undoubtedly a material to consider when selecting a storage tank for \acrshort{afff} concentrate.

\section{Areas for Further Research}
The objectives of this research work were achieved. However, the outcome of the previous studies has shown that further studies should be conducted to improve on what has been done thus far. The following avenues were recommended for future studies:

\begin{itemize}
    \item The study was limited to the impact of the three materials on any performance parameters of \acrshort{afff}. It did not focus on the causes of these effects. Further research work should be conducted to thoroughly investigate the causes of the interaction between \acrshort{afff} and mild steel, stainless steel, and \acrshort{hdpe}. In this way, it will be uncomplicated to optimize these materials in such a way that they are compatible with the \acrshort{afff} concentrate.
    \item Further research should focus on other materials that are also utilized to construct the tanks for storing \acrshort{afff} concentrate, such as fiber glass and \acrfull{xlpe}. This will establish several alternatives for storing \acrshort{afff} concentrate without causing any harm. 
    \item Further studies should focus on the fabrication methods used to construct the \acrshort{afff} concentrate storage facility. These methods may include welding, bolting, screwing, and molding. The evaluation of these methods could be an essential optimization process, as the connection joints may greatly affect the product stored in the storage tank. Besides, the connection joints may further contribute to the failure of the storage tank. 
    \item Based on the primary research, further studies should be able to investigate the precise heat treatment methods that will alter the properties of these materials, especially mild steel, as it is inexpensive. This will ensure their compatibility with \acrshort{afff} concentrate and be economical concurrently.   
\end{itemize}

\section{Conclusion}
This chapter concluded the study and suggested areas for further research. Achieving the objectives and answering the research questions fulfilled the overall aim of the study. The study utilized critical data from a theoretical and practical base to determine the impact of the storage facility on the performance parameters of \acrshort{afff}. Evaluation of the impact of mild steel, stainless steel, and \acrshort{hdpe} in the \acrshort{afff} concentrate storage tank was experimentally performed. The critical performance parameters of \acrshort{afff} affected by these materials were determined. Based on the scientific phenomena and relevant literature, the study demonstrated how these parameters influence the performance of \acrshort{afff} during firefighting circumstances in aviation fire protection.